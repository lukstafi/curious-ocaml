\documentclass{article}
\usepackage[english]{babel}
\usepackage{geometry,amsmath,theorem}
\geometry{letterpaper}

%%%%%%%%%% Start TeXmacs macros
\catcode`\>=\active \def>{
\fontencoding{T1}\selectfont\symbol{62}\fontencoding{\encodingdefault}}
\newcommand{\tmem}[1]{{\em #1\/}}
\newcommand{\tmverbatim}[1]{\text{{\ttfamily{#1}}}}
{\theorembodyfont{\rmfamily\small}\newtheorem{exercise}{Exercise}}
%%%%%%%%%% End TeXmacs macros

\begin{document}

{\class{Functional Programming}}{\titledate{February 5th 2013}}

{\title{Exam set 2}}

\begin{exercise}
  (Blue.) What is the type of the subexpression \tmverbatim{f} as part of the
  expression below assuming that the whole expression has the type given?
  
  {\hlopt{(}}{\hlkwa{fun }}{\hlstd{double g x }}{\hlopt{-> }}{\hlstd{double
  }}{\hlopt{(}}{\hlstd{g x}}{\hlopt{)) (}}{\hlkwa{fun}}{\hlstd{ f y
  }}{\hlopt{-> }}\begin{tabular}{|l|}
    \hline
    {\hlstd{f}}\\
    \hline
  \end{tabular}{\hlopt{ (}}{\hlstd{f y}}{\hlopt{))}}
  
  {\hlopt{ : (}}{\hlstd{'a }}{\hlopt{-> }}{\hlstd{'b }}{\hlopt{->
  }}{\hlstd{'b}}{\hlopt{) -> }}{\hlstd{'a }}{\hlopt{-> }}{\hlstd{'b
  }}{\hlopt{-> }}{\hlstd{'b}}
\end{exercise}

\begin{exercise}
  (Blue.) Write an example function with type:
  
  \tmverbatim{(int -> int list) -> bool}
  
  Tell ``in your words'' what it does.
\end{exercise}

\begin{exercise}
  (Green.) Find the number of elements of a list.
\end{exercise}

\begin{exercise}
  (Green.) Split a list into two parts; the length of the first part is given.
\end{exercise}

\begin{exercise}
  (Yellow.) Rotate a list N places to the left.
\end{exercise}

\begin{exercise}
  (Yellow.) Let us call a binary tree symmetric if you can draw a vertical
  line through the root node and then the right subtree is the mirror image of
  the left subtree. Write a function \tmverbatim{is\_symmetric} to check
  whether a given binary tree is symmetric.
\end{exercise}

\begin{exercise}
  (White.) By ``traverse a tree'' we mean: write a function that takes a tree
  and returns a list of values in the nodes of the tree. Traverse a tree in
  breadth-first order -- first values in more shallow nodes.
\end{exercise}

\begin{exercise}
  (White.) Generate all combinations of K distinct elements chosen from the N
  elements of a list.
\end{exercise}

\begin{exercise}
  (Orange.) Implement a topological sort of a graph: write a function that
  either returns a list of graph nodes in topological order or informs (via
  exception or option type) that the graph has a cycle.
\end{exercise}

\begin{exercise}
  (Orange.) Express \tmverbatim{fold\_left} in terms of
  \tmverbatim{fold\_right}. Hint: continuation passing style.
\end{exercise}

\begin{exercise}
  (Purple.) Show why for a monomorphic specification, if datastructures $d_1$
  and $d_2$ have the same behavior under all operations, then they have the
  same representation $d_1 = d_2$ in all implementations.
\end{exercise}

\begin{exercise}
  (Purple.) \tmverbatim{append} for lazy lists returns in constant time. Where
  has its linear-time complexity gone? Explain how you would account for this
  in a time complexity analysis.
\end{exercise}

\begin{exercise}
  (Red.) Write a function \tmverbatim{ms\_tree graph} to construct the
  {\tmem{minimal spanning tree}} of a given weighted graph. A weighted graph
  will be represented as follows:
  
  \tmverbatim{type 'a weighted\_graph = \{nodes : 'a list; edges : ('a * 'a *
  int) list\}}
  
  The labels identify the nodes \tmverbatim{'a} uniquely and there is at most
  one edge between a pair of nodes. A triple \tmverbatim{(a,b,w)} inside
  \tmverbatim{edges} corresponds to edge between \tmverbatim{a} and
  \tmverbatim{b} with weight \tmverbatim{w}. The minimal spanning tree is a
  subset of \tmverbatim{edges} that forms an undirected tree, covers all nodes
  of the graph, and has the minimal sum of weights.
\end{exercise}

\begin{exercise}
  (Crimson.) Von Koch's conjecture. Given a tree with N nodes (and hence N-1
  edges). Find a way to enumerate the nodes from 1 to N and, accordingly, the
  edges from 1 to N-1 in such a way, that for each edge K the difference of
  its node numbers equals to K. The conjecture is that this is always
  possible.
  
  For small trees the problem is easy to solve by hand. However, for larger
  trees, and 14 is already very large, it is extremely difficult to find a
  solution. And remember, we don't know for sure whether there is always a
  solution!
  
  Write a function that calculates a numbering scheme for a given tree. What
  is the solution for the larger tree pictured above?
\end{exercise}

\begin{exercise}
  (Black.) Based on our search engine implementation, write a function that
  for a list of keywords returns three best "next keyword" suggestions (in
  some sense of "best", e.g. occurring in most of documents containing the
  given words).
\end{exercise}

\

\end{document}
