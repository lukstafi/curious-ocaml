\documentclass{article}
\usepackage[english]{babel}
\usepackage{geometry,amsmath,theorem}
\geometry{letterpaper}

%%%%%%%%%% Start TeXmacs macros
\newcommand{\tmem}[1]{{\em #1\/}}
\newcommand{\tmverbatim}[1]{\text{{\ttfamily{#1}}}}
{\theorembodyfont{\rmfamily\small}\newtheorem{exercise}{Exercise}}
%%%%%%%%%% End TeXmacs macros

\begin{document}

{\class{Functional Programming}}

{\title{Compiling and Parsing}}

\begin{exercise}
  (Exercise 6.1 from {\tmem{``Modern Compiler Implementation in ML''}} by
  Andrew W. Appel.) Using the \tmverbatim{ocamlopt} compiler with parameter
  \tmverbatim{-S} and other parameters turning on all possible compiler
  optimizations, evaluate the compiled programs by these criteria:
  \begin{enumerate}
    \item Are local variables kept in registers? Show on an example.
    
    \item If local variable \tmverbatim{b} is live across more than one
    procedure call, is it kept in a callee-save register? Explain how it would
    speed up the program:\\
    {\hlkwa{let }}{\hlstd{f a }}{\hlopt{= }}{\hlkwa{let }}{\hlstd{b
    }}{\hlopt{= }}{\hlstd{a}}{\hlopt{+}}{\hlnum{1 }}{\hlkwa{in let }}{\hlstd{c
    }}{\hlopt{= }}{\hlstd{g }}{\hlopt{() }}{\hlkwa{in let }}{\hlstd{d
    }}{\hlopt{= }}{\hlstd{h c }}{\hlkwa{in
    }}{\hlstd{b}}{\hlopt{+}}{\hlstd{c}}{\hlendline{}}
    
    \item If local variable \tmverbatim{x} is never live across a procedure
    call, is it properly kept in a caller-save register? Explain how doing
    thes would speed up the program:\\
    {\hlkwa{let }}{\hlstd{h y }}{\hlopt{= }}{\hlkwa{let }}{\hlstd{x
    }}{\hlopt{= }}{\hlstd{y}}{\hlopt{+}}{\hlnum{1 }}{\hlkwa{in let }}{\hlstd{z
    }}{\hlopt{= }}{\hlstd{f y }}{\hlkwa{in }}{\hlstd{f z}}{\hlendline{}}
  \end{enumerate}
\end{exercise}

\begin{exercise}
  As above, verify whether escaping variables of a function are kept in a
  closure corresponding to the function, or in closures corresponding to the
  local, i.e. nested, functions that are returned from the function (or
  assigned to a mutable field).
\end{exercise}

\begin{exercise}
  As above, verify that OCaml compiler performs {\tmem{inline expansion}} of
  small functions. Check whether the compiler can inline, or specialize
  (produce a local function to help inlining), recursive functions.
\end{exercise}

\begin{exercise}
  Write a ``\tmverbatim{.mll} program'' that anonymizes, or masks, text. That
  is, it replaces identified probable full names (of persons, companies etc.)
  with fresh shorthands {\tmem{Mr. A}}, {\tmem{Ms. B}}, or {\tmem{Mr./Ms. C}}
  when the gender cannot be easily determined. The same (full) name should be
  replaced with the same letter.
  \begin{itemize}
    \item Do only a very rough job of course, starting with recognizing two or
    more capitalized words in a row.
  \end{itemize}
\end{exercise}

\begin{exercise}
  In the lexer {\hlkwc{EngLexer}} we call function \tmverbatim{abridged} from
  the module {\hlkwc{EngMorph}}. Inline the operation of \tmverbatim{abridged}
  into the lexer by adding a new regular expression pattern for each
  {\hlkwa{if}} clause. Assess the speedup on the {\tmem{Shakespeare}} corpus
  and the readability and either keep the change or revert it.
\end{exercise}

\begin{exercise}
  Make the lexer re-entrant for the second Menhir example (toy English grammar
  parser).
\end{exercise}

\begin{exercise}
  Make the determiner optional in the toy English grammar.
  \begin{enumerate}
    \item * Can you come up with a factorization that would avoid having two
    more productions in total?
  \end{enumerate}
\end{exercise}

\begin{exercise}
  Integrate into the {\tmem{Phrase search}} example, the {\tmem{Porter
  Stemmer}} whose source is in the \tmverbatim{stemmer.ml} file.
\end{exercise}

\begin{exercise}
  Revisit the search engine example from lecture 6.
  \begin{enumerate}
    \item Perform optimization of data structure, i.e. replace association
    lists with hash tables.
    
    \item Optimize the algorithm: perform {\tmem{query optimization}}. Measure
    time gains for selected queries.
    
    \item For bonus points, as time and interest permits, extend the query
    language with {\tmem{OR}} and {\tmem{NOT}} connectives, in addition to
    {\tmem{AND}}.
    
    \item * Extend query optimization to the query language with {\tmem{AND}},
    {\tmem{OR}} and {\tmem{NOT}} connectives.
  \end{enumerate}
\end{exercise}

\begin{exercise}
  Write an XML parser tailored to the \tmverbatim{shakespeare.xml} corpus
  provided with the phrase search example. Modify the phrase search engine to
  provide detailed information for each found location, e.g. which play and
  who speaks the phrase.
\end{exercise}

\end{document}
