\documentclass{article}
\usepackage[english]{babel}
\usepackage{geometry,hyperref}
\geometry{letterpaper}

%%%%%%%%%% Start TeXmacs macros
\newcommand{\tmem}[1]{{\em #1\/}}
\newcommand{\tmemail}[1]{\\ \textit{Email:} \texttt{#1}}
\newcommand{\tmhomepage}[1]{\\ \textit{Web:} \texttt{#1}}
\newcommand{\tmsubtitle}[1]{\thanks{\textit{Subtitle:} #1}}
%%%%%%%%%% End TeXmacs macros

\begin{document}

{\screens{{\tit{The Expression Problem}}

\title{
  The Expression Problem
  \tmsubtitle{Code organization, extensibility and reuse}
}

\author{
  {\L}ukasz Stafiniak
  \tmemail{lukstafi@gmail.com}
  \tmhomepage{www.ii.uni.wroc.pl/\~{}lukstafi}
}

\maketitle
\begin{itemize}
  \item Ralf L{\"a}mmel lectures on MSDN's Channel 9:\\
  \href{http://channel9.msdn.com/Shows/Going+Deep/C9-Lectures-Dr-Ralf-Laemmel-Advanced-Functional-Programming-The-Expression-Problem}{The
  Expression Problem},
  \href{http://channel9.msdn.com/Shows/Going+Deep/C9-Lectures-Dr-Ralf-Lmmel-Advanced-Functional-Programming-Type-Classes}{Haskell's
  Type Classes}
  
  \item The old book {\tmem{Developing Applications with Objective Caml}}:\\
  \href{http://caml.inria.fr/pub/docs/oreilly-book/html/book-ora153.html}{Comparison
  of Modules and Objects},
  \href{http://caml.inria.fr/pub/docs/oreilly-book/html/book-ora154.html}{Extending
  Components}
  
  \item The new book {\tmem{Real World OCaml}}:
  \href{https://realworldocaml.org/v1/en/html/objects.html}{Chapter 11:
  Objects}, \href{https://realworldocaml.org/v1/en/html/classes.html}{Chapter
  12: Classes}
  
  \item Jacques Garrigue's
  \href{http://www.math.nagoya-u.ac.jp/~garrigue/papers/variant-reuse.ps.gz}{Code
  reuse through polymorphic variants},\\
  and
  \href{http://www.math.nagoya-u.ac.jp/~garrigue/papers/nakata-icfp2006.pdf}{Recursive
  Modules for Programming} with Keiko Nakata
  
  \item
  \href{http://caml.inria.fr/pub/docs/manual-ocaml/extn.html#sec246}{Extensible
  variant types}
  
  \item Graham Hutton's and Erik Meijer's
  \href{https://www.cs.nott.ac.uk/~gmh/monparsing.pdf}{Monadic Parser
  Combinators}
\end{itemize}}{}{}{}{}{}{}{}{}{}{}{}{}{}{}{}{}{}}

\end{document}
