\documentclass{article}
\usepackage[english]{babel}
\usepackage{geometry,amsmath,theorem}
\geometry{letterpaper}

%%%%%%%%%% Start TeXmacs macros
\catcode`\>=\active \def>{
\fontencoding{T1}\selectfont\symbol{62}\fontencoding{\encodingdefault}}
\newcommand{\tmem}[1]{{\em #1\/}}
\newcommand{\tmop}[1]{\ensuremath{\operatorname{#1}}}
\newcommand{\tmverbatim}[1]{\text{{\ttfamily{#1}}}}
{\theorembodyfont{\rmfamily\small}\newtheorem{exercise}{Exercise}}
%%%%%%%%%% End TeXmacs macros

\begin{document}

{\class{Functional Programming}}{\titledate{February 5th 2013}}

{\title{Exam set 1}}

\begin{exercise}
  (Blue.) What is the type of the subexpression \tmverbatim{y} as part of the
  expression below assuming that the whole expression has the type given?
  
  {\hlopt{(}}{\hlkwa{fun }}{\hlstd{double g x }}{\hlopt{-> }}{\hlstd{double
  }}{\hlopt{(}}{\hlstd{g x}}{\hlopt{))}}{\hlstd{}}
  {\hlopt{(}}{\hlkwa{fun}}{\hlstd{ f y }}{\hlopt{-> }}{\hlstd{f
  }}{\hlopt{(}}{\hlstd{f \begin{tabular}{|l|}
    \hline
    y\\
    \hline
  \end{tabular}}}{\hlopt{))}}
  
  {\hlopt{ : (}}{\hlstd{'a }}{\hlopt{-> }}{\hlstd{'b }}{\hlopt{->
  }}{\hlstd{'b}}{\hlopt{) -> }}{\hlstd{'a }}{\hlopt{-> }}{\hlstd{'b
  }}{\hlopt{-> }}{\hlstd{'b}}
\end{exercise}

\begin{exercise}
  (Blue.) Write an example function with type:
  
  \tmverbatim{((int -> int) -> bool) -> int}
  
  Tell ``in your words'' what it does.
\end{exercise}

\begin{exercise}
  (Green.) Write a function \tmverbatim{last : 'a list -> 'a option} that
  returns the last element of a list.
\end{exercise}

\begin{exercise}
  (Green.) Duplicate the elements of a list.
\end{exercise}

\begin{exercise}
  (Yellow.) Drop every N'th element from a list.
\end{exercise}

\begin{exercise}
  (Yellow.) Construct completely balanced binary trees of given depth.
  
  In a completely balanced binary tree, the following property holds for every
  node: The number of nodes in its left subtree and the number of nodes in its
  right subtree are almost equal, which means their difference is not greater
  than one.
  
  Write a function \tmverbatim{cbal\_tree} to construct completely balanced
  binary trees for a given number of nodes. The function should generate the
  list of all solutions (e.g. via backtracking). Put the letter
  \tmverbatim{'x'} as information into all nodes of the tree.
\end{exercise}

\begin{exercise}
  (White.) Due to Yaron Minsky.
  
  Consider a datatype to store internet connection information. The time
  \tmverbatim{when\_initiated} marks the start of connecting and is not needed
  after the connection is established (it is only used to decide whether to
  give up trying to connect). The ping information is available for
  established connection but not straight away.
  
  \\
  {\hlstd{}}{\hlkwa{type }}{\hlstd{connection{\textunderscore}state
  }}{\hlopt{=}}{\hlendline{}}\\
  {\hlstd{}}{\hlstd{ \ }}{\hlopt{\textbar
  }}{\hlkwd{Connecting}}{\hlendline{}}\\
  {\hlstd{}}{\hlstd{ \ }}{\hlopt{\textbar
  }}{\hlkwd{Connected}}{\hlendline{}}\\
  {\hlstd{}}{\hlstd{ \ }}{\hlopt{\textbar
  }}{\hlkwd{Disconnected}}{\hlendline{}}\\
  {\hlstd{}}{\hlendline{}}\\
  {\hlkwa{type }}{\hlstd{connection{\textunderscore}info }}{\hlopt{=
  \{}}{\hlendline{}}\\
  {\hlstd{}}{\hlstd{ \ }}{\hlstd{state }}{\hlopt{:
  }}{\hlstd{connection{\textunderscore}state}}{\hlopt{;}}{\hlendline{}}\\
  {\hlstd{}}{\hlstd{ \ }}{\hlstd{server }}{\hlopt{:
  }}{\hlstd{}}{\hlkwc{Inet{\textunderscore}addr}}{\hlstd{}}{\hlopt{.}}{\hlstd{t}}{\hlopt{;}}{\hlendline{}}\\
  {\hlstd{}}{\hlstd{ \
  }}{\hlstd{last{\textunderscore}ping{\textunderscore}time }}{\hlopt{:
  }}{\hlstd{}}{\hlkwc{Time}}{\hlstd{}}{\hlopt{.}}{\hlstd{t
  }}{\hlkwb{option}}{\hlstd{}}{\hlopt{;}}{\hlendline{}}\\
  {\hlstd{}}{\hlstd{ \ }}{\hlstd{last{\textunderscore}ping{\textunderscore}id
  }}{\hlopt{: }}{\hlstd{}}{\hlkwb{int
  option}}{\hlstd{}}{\hlopt{;}}{\hlendline{}}\\
  {\hlstd{}}{\hlstd{ \ }}{\hlstd{session{\textunderscore}id }}{\hlopt{:
  }}{\hlstd{}}{\hlkwb{string option}}{\hlstd{}}{\hlopt{;}}{\hlendline{}}\\
  {\hlstd{}}{\hlstd{ \ }}{\hlstd{when{\textunderscore}initiated }}{\hlopt{:
  }}{\hlstd{}}{\hlkwc{Time}}{\hlstd{}}{\hlopt{.}}{\hlstd{t
  }}{\hlkwb{option}}{\hlstd{}}{\hlopt{;}}{\hlendline{}}\\
  {\hlstd{}}{\hlstd{ \ }}{\hlstd{when{\textunderscore}disconnected }}{\hlopt{:
  }}{\hlstd{}}{\hlkwc{Time}}{\hlstd{}}{\hlopt{.}}{\hlstd{t
  }}{\hlkwb{option}}{\hlstd{}}{\hlopt{;}}{\hlendline{}}\\
  {\hlstd{}}{\hlopt{\}}}{\hlstd{}}{\hlendline{}}
  
  (The types {\hlkwc{Time}}{\hlstd{}}{\hlopt{.}}{\hlstd{t }}and
  {\hlkwc{Inet{\textunderscore}addr}}{\hlstd{}}{\hlopt{.}}{\hlstd{t}} come
  from the library {\tmem{Core}} used where Yaron Minsky works. You can
  replace them with \tmverbatim{float} and
  {\hlkwc{Unix}}{\hlstd{}}{\hlopt{.}}{\hlstd{inet\_addr}}. Load the Unix
  library in the interactive toplevel by \tmverbatim{\#load "unix.cma";;}.)
  Rewrite the type definitions so that the datatype will contain only
  reasonable combinations of information.
\end{exercise}

\begin{exercise}
  (White.) Design an algebraic specification and write a signature for
  first-in-first-out queues. Provide two implementations: one straightforward
  using a list, and another one using two lists: one for freshly added
  elements providing efficient queueing of new elements, and ``reversed'' one
  for efficient popping of old elements.
\end{exercise}

\begin{exercise}
  (Orange.) Implement \tmverbatim{while\_do} in terms of
  \tmverbatim{repeat\_until}.
\end{exercise}

\begin{exercise}
  (Orange.) Implement a map from keys to values (a dictionary) using only
  functions (without data structures like lists or trees).
\end{exercise}

\begin{exercise}
  (Purple.) One way to express constraints on a polymorphic function is to
  write its type in the form: $\forall \alpha_1 \ldots \alpha_n [C] . \tau$,
  where $\tau$ is the type of the function, $\alpha_1 \ldots \alpha_n$ are the
  polymorphic type variables, and $C$ are additional constraints that the
  variables $\alpha_1 \ldots \alpha_n$ have to meet. Let's say we allow
  ``local variables'' in $C$: for example $C = \exists \beta . \alpha_1
  \dot{=} \tmop{list} (\beta)$. Why the general form $\forall \beta [C] .
  \beta$ is enough to express all types of the general form $\forall \alpha_1
  \ldots \alpha_n [C] . \tau$?
\end{exercise}

\begin{exercise}
  (Purple.) Define a type that corresponds to a set with a googleplex of
  elements (i.e. $10^{10^{100}}$ elements).
\end{exercise}

\begin{exercise}
  (Red.) In a height-balanced binary tree, the following property holds for
  every node: The height of its left subtree and the height of its right
  subtree are almost equal, which means their difference is not greater than
  one. Consider a height-balanced binary tree of height $h$. What is the
  maximum number of nodes it can contain? Clearly, $\tmop{maxN} = 2 h - 1$.
  However, finding the minimum number $\tmop{minN}$ is more difficult.
  
  Construct all the height-balanced binary trees with a given nuber of nodes.
  \tmverbatim{hbal\_tree\_nodes n} returns a list of all height-balanced
  binary tree with \tmverbatim{n} nodes.
  
  Find out how many height-balanced trees exist for \tmverbatim{n} = 15.
\end{exercise}

\begin{exercise}
  (Crimson.) To construct a Huffman code for symbols with
  probability/frequency, we can start by building a binary tree as follows.
  The algorithm uses a priority queue where the node with lowest probability
  is given highest priority:
  \begin{enumerate}
    \item Create a leaf node for each symbol and add it to the priority queue.
    
    \item While there is more than one node in the queue:
    \begin{enumerate}
      \item Remove the two nodes of highest priority (lowest probability) from
      the queue.
      
      \item Create a new internal node with these two nodes as children and
      with probability equal to the sum of the two nodes' probabilities.
      
      \item Add the new node to the queue.
    \end{enumerate}
    \item The remaining node is the root node and the tree is complete.
  \end{enumerate}
  Label each left edge by \tmverbatim{0} and right edge by \tmverbatim{1}. The
  final binary code assigns the string of bits on the path from root to the
  symbol as its code.
  
  We suppose a set of symbols with their frequencies, given as a list of
  Fr(S,F) terms. Example: \tmverbatim{fs = [Fr(a,45); Fr(b,13); Fr(c,12);
  Fr(d,16); Fr(e,9); Fr(f,5)]}. Our objective is to construct a list
  \tmverbatim{Hc(S,C)} terms, where \tmverbatim{C} is the Huffman code word
  for the symbol \tmverbatim{S}. In our example, the result could be
  \tmverbatim{hs = [Hc(a,'0'); Hc(b,'101'); Hc(c,'100'); Hc(d,'111');
  Hc(e,'1101'); Hc(f,'1100')]} [\tmverbatim{Hc(a,'01')},...etc.]. The task
  shall be performed by the function huffman defined as follows:
  \tmverbatim{huffman(fs)} returns the Huffman code table for the frequency
  table \tmverbatim{fs}.
\end{exercise}

\begin{exercise}
  (Black.) Implement the Gaussian Elimination algorithm for solving linear
  equations and inverting square invertible matrices.
\end{exercise}

\end{document}
